\documentclass{report}
    \linespread{1.5}
    
    \usepackage{amsmath}
    \usepackage{amssymb}
    
    
    \begin{document}
    CISC102 - Discrete Math I
    
    July 2018
    
    \bigskip
        
    Assignment 3 - Problem 7
    
    Sam Huang (10175794)
    \bigskip
    
    \centerline{\textbf{Part A}}
    
    \centerline{\textbf{Prove that}} 
    
    $$GCD(m*a,m*b)=m*GCD(a,b)$$
    
    \bigskip
    
    \textbf{Direct proof, starting from the LHS:}
    
    Let  $u = GCD(m*a, m*b)$
    
    \textit{u can be expressed as u = ax + by  for integers x and y, given}
    
    \textit{Theorem 11.13 from the textbook, which states that if d is the} 
    
    \textit{smallest positive integer of the form a*x+b*y, then d = GCD(a,b)}
    
    $u = (m*a)*x + (m*b)*y$
    
    Therefore, u is a sum of multiples of m
    
    $ m | u$
    
    Therefore, u is a multiple of m
    
    $ u = p*m$, for some integer p
    
    \textit{Substitute $u = p*m$ in $u = (m*a)*x + (m*b)*y$}
    
    $p*m = (m*a)*x + (m*b)*y$
    
    \bigskip
    
    
    To prove the equality $u = GCD(m*a, m*b) = m*GCD(a,b) = m*p$, we 
    
    need to show that $p = d = GCD(a,b)$
    
    To prove $p = d$, we need to show that $p \geq d$ and $d \geq p$
    
    \medskip
    
    \textbf{To show $p \geq d$:}
    
    Theorem 11.13 tells us that d is the smallest positive integer which can
    
    be expressed as an linear combination of a and b where a and b are
    
    integers. 
    
    \textit{We know that}
    
    $\rightarrow p*m = (m*a)*x + (m*b)*y$
    
    \textit{Divide out m}
    
    $\rightarrow p = a*x + b*y$
    
    Since p is another such linear expression of a and b, d cannot
    
    be greater than p. Therefore, $p \geq d$
    
    \medskip
    
    \textbf{To show $d \geq p$:}
    
    $d = ax + by$
    
    \textit{Multiply both sides by m}
    
    $\rightarrow m*d = (m*a)*x + (m*b)*y$
    
    Since u is the smallest positive integer of the form $a*x+b*y$ for $a=m*a$
    
    and $b=m*b$, then $m*d \geq u$
    
    $\rightarrow m*d \geq p*m$
    
    $\rightarrow d \geq p$
    
    \bigskip
    
    Now we have shown that $p \geq d$ and $d \geq p$, we know that $p = d$.
    
    We can substitute $p=d$ back into the equality we need to prove
    
    $u = GCD(m*a, m*b) = m*GCD(a,b) = m*p = m*d = m*GCD(a,b)$
    
    \medskip
    
    Therefore, we have proven that the theorem $$GCD(m*a,m*b)=m*GCD(a,b)$$ 
    
    is true. 
    $\blacksquare$
    
    \bigskip
    
    \centerline{\textbf{Part B}}
    
    \centerline{\textbf{Prove that}} 
    
    \medskip
    
    \centerline{If $GCD(a,m) = d$ and $GCD(b,m) = 1$, then $GCD(a*b,m) = d$}
    
    \bigskip
    
    \textbf{Direct proof}
    
    Given $GCD(a,m) = d$, we know that: $d | a$ and $d | m$
    
    Therefore, 
    
    $a = x*d$,
    $m = y*d$,
    $b = z*d + r$ 
    
    for integers x,y,z
    
    \textit{Multiplying a and b, we get}
    
    $a*b = x*d*(z*d+r)$
    
    \textit{Expand RHS}
    
    $\rightarrow a*b = x*d*z*d + r*x*d$
    
    Let some integers $n_{1} = x*d*z$ and $n_{2} = r*x$
    
    $\rightarrow a*b = n_{1}*d + n_{2}*d$
    
    $\rightarrow a*b = d*(n_{1} + n_{2})$
    
    We see that $a*b$ is the sum of multiples of d, therefore $d | (a*b)$
    
    \medskip
    
    We now know: 
    
    $d | (a*b)$,
    
    $d | a$ and $d \nmid b$,
    
    $d | m$ and $GCD(b,m) = 1$, which means b and m are relatively prime
    
    Therefore, $GCD(a*b, m) = d$. We have proven that the theorem is true. $\blacksquare$
    
    \end{document}
    
        